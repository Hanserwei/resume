\documentclass{resume}
\usepackage{zh_CN-Adobefonts_external} % Simplified Chinese Support using external fonts (./fonts/zh_CN-Adobe/)
%\usepackage{zh_CN-Adobefonts_internal} % Simplified Chinese Support using system fonts
\usepackage{linespacing_fix} % disable extra space before next section
\usepackage{cite}
\usepackage{ragged2e} % For Justified alignment
\usepackage{amsmath} % For \text command in math mode

\begin{document}
\pagenumbering{gobble} % suppress displaying page number

% 个人信息和定位
\name{伍维}
\tagline{\textbf{求职意向:Java 后端开发工程师} \quad 离职-随时到岗}

\basicInfo{
  \email{hanserwei@qq.com} \textperiodcentered\ 
  \phone{17628516882} \textperiodcentered\ 
  \faLink\ \url{https://likeyy.love/} \textperiodcentered\ 
  \faGithub\ \url{https://github.com/hanserwei} \textperiodcentered\ 
  \faMapMarker\ 成都
  }

\section{个人简介}
\begin{itemize}[parsep=0.4ex]
  \item 2 年后端研发经验,熟悉 Java 21 及虚拟线程,能从 0-1 设计并交付高并发、高可用的微服务系统。
  \item 具备从需求拆解、数据建模、接口设计到性能压测与优化的完整经验,善用缓存、异步与消息队列提升吞吐与可靠性。
  \item 熟悉全链路工具链:日志与指标监控,能独立完成生产环境部署与稳定性治理。
\end{itemize}
 
\section{\texorpdfstring{\faCogs\ }{}专业技能}
\begin{itemize}[parsep=0.5ex]
  \item \textbf{语言与底层:} Java 8/17/21,熟悉虚拟线程、集合与并发包原理,理解 JVM 内存模型与常见 GC。
  \item \textbf{框架与架构:} Spring Boot / Cloud Alibaba / MVC,MyBatis-Plus、JPA;掌握 IOC、AOP、服务注册配置、限流与熔断。
  \item \textbf{存储与缓存:} MySQL/PG(索引、事务、MVCC)、Redis(数据结构、持久化、过期/淘汰策略),Caffeine + Redis 二级缓存。
  \item \textbf{消息与异步:} RocketMQ / RabbitMQ,熟悉消息可靠性、幂等、顺序与积压处理。
  \item \textbf{工程与运维:} Linux、Docker、Nginx;掌握日志采集、指标监控与告警的基本建设。
  \item \textbf{前端与协作:} Vue3/TypeScript 可独立完成管理端页面。
\end{itemize}

\section{\texorpdfstring{\faUsers\ }{}工作经历}
\datedsubsection{\textbf{国电南自}, 南京}{2024年7月 -- 2025年7月}
\role{研发部}{研发工程师}
\begin{itemize}
  \item 主导智慧电厂监测平台不停服迁移重构:设计“双写+灰度切流”策略,批量迁移 1700W+ 告警日志至分库分表集群,迁移期业务零中断,核心查询性能提升约 5 倍。
  \item 负责电力交易与调度数据统一报送系统(单体):引入分库分表与 RabbitMQ 削峰异步写入,设计幂等防重、重试与死信补偿机制,保障高峰期写入吞吐与报送到达率。
\end{itemize}

\section{\texorpdfstring{\faGraduationCap\ }{}教育背景}
\datedsubsection{\textbf{西南交通大学} 211}{2020年9月 -- 2024年6月}
\textit{本科}\ 电气工程与智能控制 (全日制)

\section{\texorpdfstring{\faCode\ }{}项目经历}

\datedsubsection{\textbf{智慧电厂分布式设备监测数据平台(不停服迁移重构)}, 南京}{2024年8月 -- 2025年6月}
\role{Java 后端开发工程师}{不停服迁移与分库分表 | 技术栈:Spring Boot, Spring Cloud Alibaba, MySQL 分库分表, RabbitMQ, Redis, Canal}
\begin{onehalfspacing}
\begin{itemize}
  \item \textbf{不停服迁移方案:} 设计“双写+双读+灰度切流”,记录时间点 $T$ 区分存量/增量,离线 Task 分批搬迁 \textbf{1700W+ 告警日志} 至分库分表集群,迁移期业务零中断。
  \item \textbf{异步双写与补偿:} 采集接口末端接入自定义线程池异步写新库,失败数据降级投递 RabbitMQ 补偿/死信重试,保证最终一致性。
  \item \textbf{一致性校验与修复:} Canal 订阅新旧库 Binlog 做 \textbf{1s 延迟校验},发现不一致自动以旧库为准修正,新库切读后核心查询性能提升约 5 倍。
  \item \textbf{灰度切读与防重:} Nacos 配置动态控制读流量阶梯切换($1\%\rightarrow 5\%\rightarrow 20\%\rightarrow 50\%\rightarrow 100\%$),Redis 分布式锁防重校验,异常时秒级回滚。
\end{itemize}
\end{onehalfspacing}

\datedsubsection{\textbf{电力交易与调度数据统一报送系统(单体架构)}, 南京}{2024年 -- 2025年}
\role{Java 后端开发工程师}{高吞吐采集与报送网关 | 技术栈:Spring Boot, MySQL 分库分表, RabbitMQ, Redis, XXL-JOB, MyBatis-Plus}
\begin{onehalfspacing}
面向省区节点部署的高性能单体应用,负责场站发电交易/结算数据的采集清洗、持久化写入分库分表集群,并向调度中心可靠上送。
\begin{itemize}
  \item \textbf{分库分表落地:} 负责报送明细表水平拆分与路由实现(按 \texttt{station\_id} 取模分片),结合 MyBatis-Plus 动态表名能力完成 SQL 自动改写,缓解单表写入瓶颈。
  \item \textbf{削峰与异步写入:} 基于 RabbitMQ 设计“生产-消费”异步链路,平滑早晚高峰流量冲击;消费端仅在写入成功后 \textbf{Manual Ack},失败按策略重试并进入 DLQ 兜底。
  \item \textbf{幂等与防重:} 在库表建立联合唯一约束(\texttt{station\_id, trade\_no}),并在入口侧引入 Redis 短时缓存/布隆过滤器预判,降低重复报文对数据库的冲击。
  \item \textbf{稳定性与补偿:} 使用 XXL-JOB 分片广播在低峰期并行汇总上报;外部接口超时触发降级,将报文转存 MQ 异步推送,避免外部故障占满业务线程池。
\end{itemize}
\end{onehalfspacing}

\datedsubsection{\textbf{AI全流程模拟面试平台 (因AI面试)}, 成都}{2025年10月 -- 2025年11月}
\role{后端开发兼前端开发}{技术栈:Spring AI, Java 21, pgvector, WebSocket\\ 体验地址:\url{https://interview.likeyy.love/}}
\begin{onehalfspacing}
生产级全栈后端项目,集成多个大模型能力和实时音频处理,实现从简历解析到报告生成的完整 AI 面试闭环。
\begin{itemize}
  \item \textbf{多模型实时交互:} 负责平台架构设计,\textbf{集成 DashScope 大模型}、\textbf{火山 ASR} 和 \textbf{TTS} 服务,实现面试全流程和实时语音交互。
  \item \textbf{动态对话决策:} 实现 AI 面试官对回答质量的实时评估与追问/跳题决策,平均面试时长缩短 20\%,交互更贴合候选人水平。
  \item \textbf{并发与流式处理:} JDK 21 虚拟线程并行生成评估报告,STOMP WebSocket 推送状态(QUEUED $\rightarrow$ RUNNING $\rightarrow$ SUCCESS),端到端延迟下降 35\%。
  \item \textbf{RAG智能推荐:} PostgreSQL + pgvector 存储题目向量,Spring AI 检索生成推荐题单;基于报告薄弱项做相似度召回,匹配度提升显著。
\end{itemize}
\end{onehalfspacing}

\datedsubsection{\textbf{Han-note (仿小红书社区项目)}, 成都}{2025年7月 -- 2025年9月}
\role{Java 后端开发工程师}{微服务分布式架构 | 技术栈:Spring Cloud Alibaba, RocketMQ, Redis, Elasticsearch, xxl-job, Canal\\ 仓库:\url{https://gitea.likeyy.love/Hanserwei/han-note}}
\begin{onehalfspacing}
基于微服务分布式架构,设计并从0落地高并发、高可用的仿小红书社区平台,支撑海量用户读写和数据一致性要求。
\begin{itemize}
  \item \textbf{高并发写处理:} 针对点赞、关注等高频写操作,引入 Redis Roaring Bitmap 高性能判断用户状态,结合 RocketMQ 异步落库和 RateLimiter 令牌桶进行流量削峰,保障数据库稳定。
  \item \textbf{高性能读优化:} Redis + Caffeine 二级缓存支撑笔记详情/用户信息的\textbf{高并发读请求},设计防击穿/穿透/雪崩策略(预热、空值、随机过期)。
  \item \textbf{分布式与一致性:} 接入美团 Leaf 分布式 ID,单节点吞吐量达 $\text{22000+}/\text{s}$;RocketMQ 事件驱动保障集群缓存最终一致性,幂等由去重表 + 业务唯一键控制。
  \item \textbf{实时搜索与监控:} Canal 监听 Binlog 增量同步至 Elasticsearch,笔记/用户索引延迟控制在秒级;接入 Prometheus + Grafana 监控 QPS/RT/失败率。
\end{itemize}
\end{onehalfspacing}

\section{\texorpdfstring{\faHeartO\ }{}荣誉奖项}
\datedline{CET-6, 国家励志奖学金,学院奖学金}{2020-2023}
\datedline{优秀新员工}{2025}

\end{document}
