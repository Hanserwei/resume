% !TEX TS-program = xelatex
% !TEX encoding = UTF-8 Unicode
% !Mode:: "TeX:UTF-8"

\documentclass{resume}
\usepackage{zh_CN-Adobefonts_external} % Simplified Chinese Support using external fonts (./fonts/zh_CN-Adobe/)
%\usepackage{zh_CN-Adobefonts_internal} % Simplified Chinese Support using system fonts
\usepackage{linespacing_fix} % disable extra space before next section
\usepackage{cite}
\usepackage{ragged2e} % For Justified alignment
\usepackage{amsmath} % For \text command in math mode

\begin{document}
\pagenumbering{gobble} % suppress displaying page number

% 个人信息和定位
\name{伍维 \quad\quad\quad \small{\textbf{求职意向:Java 后端开发工程师} \quad \small{离职-随时到岗}}}

\basicInfo{
  \email{hanserwei@qq.com} \textperiodcentered\ 
  \phone{17628516882} \textperiodcentered\ 
  \faLink\ \url{https://likeyy.love/}
  }
 
\section{\faCogs\ 专业技能}
\begin{itemize}[parsep=0.5ex]
  \item 熟练掌握Java基础,熟悉Java21的虚拟线程等新特性,了解常用集合底层原理,熟悉JUC并发编程,对synchronized,volatile等底层有过了解,了解JVM(内存加载机制、JVM内存模型、常见的垃圾回收机制);
  \item 熟练使用Spring Boot,Spring Cloud Alibaba、Spring MVC,MyBatis,MyBatis-plus,JPA等常见框架,熟悉Spring IOC、Spring AOP的原理,了解Spring Cloud等分布式架构;
  \item 熟悉MySQL数据库的使用,了解其基本架构(MySQL索引、事务、MVCC等),掌握PostgreSQL数据库的基本使用;
  \item 熟悉Redis数据库,了解其基本数据类型和对应的应用场景,了解Redis持久化和过期删除策略等;
  \item 了解RocketMQ消息队列的使用,掌握对消息可靠性、幂等性和消息积压等的处理;
  \item 熟悉HTML、CSS、JavaScript、TypeScript以及Vue3,可以完成一般前端页面的开发;
  \item 熟练使用Linux常用命令,拥有使用Docker从零到一容器化部署项目的能力。
\end{itemize}

\section{\faUsers\ 工作经历}
\datedsubsection{\textbf{国电南自}, 南京}{2024年7月 -- 2025年7月}
\role{研发部}{研发工程师}
\begin{itemize}
  \item 负责公司内部系统的开发和维护,参与需求分析、数据库设计、接口开发等工作,主要使用 Java、Spring Boot、MySQL 等技术。
  \item 参与搭建公司 RAG 知识库以及本地大模型部署的相关工作,负责知识库服务的技术选型、服务搭建与需求设计。
\end{itemize}

\section{\faGraduationCap\ 教育背景}
\datedsubsection{\textbf{西南交通大学} 211}{2020年9月 -- 2024年6月}
\textit{本科}\ 电气工程 (全日制)

\section{\faHeartO\ 荣誉奖项}
\datedline{CET-6, 国家励志奖学金,学院奖学金}{2020-2023}

\section{\faCode\ 项目经历}
\datedsubsection{\textbf{Han-note (仿小红书社区项目)}, 成都}{2025年7月 -- 2025年9月}
\role{Java 后端开发工程师}{微服务分布式架构 | 技术栈:Spring Cloud Alibaba, RocketMQ, Redis, Elasticsearch}
\begin{onehalfspacing}
基于微服务分布式架构,设计并从0落地高并发、高可用的仿小红书社区平台,支撑海量用户读写和数据一致性要求。
\begin{itemize}
  \item \textbf{高并发写处理:} 针对点赞、关注等高频写操作,引入 Redis Roaring Bitmap 高性能判断用户状态,结合 RocketMQ 异步落库和 RateLimiter 令牌桶进行流量削峰,保障数据库稳定。
  \item \textbf{高性能读优化:} 通过 Redis + Caffeine 本地缓存构建二级缓存,有效支撑笔记详情、用户信息查询的\textbf{高并发读请求},成功解决了缓存雪崩/穿透/击穿问题。
  \item \textbf{分布式与一致性:} 整合并部署美团 Leaf 分布式 ID 服务,单节点吞吐量达 $\text{22000+}/\text{s}$,日承载 $\text{19}$ 亿+次 ID 生成。设计基于 RocketMQ 的消息驱动机制,保障集群环境下缓存的最终一致性。
  \item \textbf{实时搜索:} 基于 Canal 实时监听 MySQL Binlog,自定义事件处理器实时同步增量数据至 Elasticsearch,实现笔记/用户搜索的\textbf{毫秒级实时索引构建}。
\end{itemize}
\end{onehalfspacing}

\datedsubsection{\textbf{AI全流程模拟面试平台 (因AI面试)}, 成都}{2025年10月 -- 2025年11月}
\role{后端开发兼前端开发}{技术栈:Spring AI, Java 21, pgvector, WebSocket}
\begin{onehalfspacing}
生产级全栈后端项目,集成多个大模型能力和实时音视频处理,实现从简历解析到报告生成的完整 AI 面试闭环。
\begin{itemize}
  \item \textbf{多模型实时交互:} 负责平台架构设计,\textbf{集成 DashScope 大模型}、\textbf{火山 ASR} 和 \textbf{TTS} 服务,实现面试全流程和实时语音交互。
  \item \textbf{动态对话决策:} 实现 AI 智能面试官的动态交互逻辑,根据候选人回答质量,自动判断执行追问、转入下一题或结束面试的决策。
  \item \textbf{并发与流式处理:} 引入 \textbf{JDK 21 虚拟线程} 异步生成复杂评估报告,并利用 \textbf{STOMP WebSocket} 实时推送报告进度(QUEUED $\rightarrow$ SUCCESS)。
  \item \textbf{RAG智能推荐:} 利用 \textbf{PostgreSQL 的 pgvector 扩展} 和 Spring AI pgvector-store,实现题目向量存储和检索。基于报告中的薄弱点,\textbf{进行相似度搜索},返回个性化练习题库。
\end{itemize}
\end{onehalfspacing}

\end{document}